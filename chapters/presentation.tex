%!TEX program=luatex

\newpage
\section{Présentation de l'application "Feedny"}
Après avoir finaliser les différents modules qui composent notre système, nous avons procéder à l'intégration de ces derniers dans une application mobile multi plate-formes (Androïd et IOS) dans l'optique de permettre à l'utilisateur de bénéficier de chaque fonctionnalité.!!

% L'expérience utilisateur au sein de notre application est étudier pour optimiser la navigation entre les différents espaces. L'application est très intuitifs et conçu afin d'attirer les utilisateurs le plus de temps possible.
La phase d'implémentation a été divisée en deux parties, le développement Back-end de l'API qui est la pièce motrice de l'application et qui intègre tout les modèles de classification, résumé et recommandation, et le développement de l'interface utilisateur sous forme d'une application mobile.

\begin{figure}[H]
    \centering
    \includegraphics[height=132pt,width=270pt]{img/chapter4/frontbackend.png}
    \caption{API et application mobile - les deux composants principales de \textquotedbl Feedny\textquotedbl }
    \label{frontbackend}
\end{figure}

Nous allons présenter chaque composant apparu sur la \autoref{frontbackend} et ses étapes d'implémentations en détails dans ce qui suit.

    \subsection{Développement Back-end}
    La partie "métier" de l'application consiste en un service web sous l'architecture REST\footnote{Respresentational State Transfert : un style d'architecture définissant un ensemble de contraintes et de propriétés basées sur le protocole HTTP [Wikipédia]} API\footnote{Application Programming Interface : un ensemble normalisé de classes, de méthodes ou de fonctions qui sert de façade par laquelle un logiciel offre des services à d'autres logiciels [Wikipédia]} qui intègre tout les modules. L'API reçoit les requêtes des utilisateurs à travers l'interface, et retourne une réponse via le protocole HTTP.

    Le Back-end est développé entièrement en Python en utilisant le Framework Flask\autoref{flask}. L'API est hébergée, pendant la phase de développement, dans un serveur web. 
    
    Elle est composée de deux modules :
        \begin{itemize}
            \item Module d'extraction d'articles
            \item Module de gestion de la base de données
        \end{itemize} 

        \subsubsection{Extraction d'articles}
        Le composant le plus important dans notre système est l'article de presse. Ce dernier se trouve un peu partout sur les revues de presses en ligne. À cet effet, nous avons implémenter une fonctionnalité qui permet d'extraire des articles de presse de différentes sources dans les deux langues.

        L'extraction d'articles de presse peut être personnalisée selon le besoin, on peut choisir des articles à partir des catégories, région, pays, etc.

        Lors de l'extraction d'un article, nous récupérons toutes les informations relatives à ce dernier tels que : le contenu de l'article, le titre, l'auteur, l'horaire de publication, etc.

        Le module d'extraction d'article de presse peut être invoquer juste en saisissant une requête dans le navigateur ou en visitant la page principale de notre application, la réponse est retournée sous le format JSON, comme dans l'exemple suivant qui monte une extraction par nom de source :

\begin{lstlisting}[style=api] 
  http://feedny.io/api/articles/add/sources=al-jazeera-english
\end{lstlisting}
        
        ou encore par catégorie :
\begin{lstlisting}[style=api] 
  http://feedny.io/api/articles/add/categories=sport,health
\end{lstlisting}  

        \subsubsection{Gestion de la base de données}
        Le module de gestion de la base de données s'occupe de l'insertion, la suppression, la recherche et la mise à jour des articles et des profils utilisateurs.

        Chaque appel à l'API implique systématiquement une opération implicite sur la base de données. Elle est composée, comme cité dans \autoref{}, de deux collections : Articles et Profils. 

        \begin{enumerate}[leftmargin=*]
            \item\textbf{Gestion de la collection d'Articles}\\
            Après extraction d'un article de presse, il sera stocké dans un document qui apparient à la collection Articles dans notre base de données NoSQL afin de faciliter la recherche et de remédier à la lenteur du débit internet. (la structure d'un document Article a été présentée dans \autoref{})

            La recherche peut se faire avec n'importe quel attribut d'un article de presse, et c'est l'un des plus grand avantages des base de données NoSQL. Les articles peuvent aussi avoir une structure différentes, et on peut stocké, également, des images et du son, s'il y-en a.

            On peut également proposer des articles à un utilisateur, en utilisant son vecteur de probabilité de sélection et ses sources préférées : 
\begin{lstlisting}[style=api] 
  http://feedny.io/api/profiles/onload/username=yankheloufi
\end{lstlisting} 
            
            \item\textbf{Gestion de la collection de Profils}\\
            La structure des profils utilisateurs est très dynamique, elle est différentes d'un utilisateur à un autre. Les utilisateurs peuvent avoir plusieurs préférences et source favorites. La gestion des profils est également effectué à partir de l'API, un nouvel utilisateur est ajouter de la façon suivante :   
\begin{lstlisting}[style=api] 
  http://feedny.io/api/profiles/add/profile=username::password::user@hey.com::sport,religion::bbc-news,echourouk
\end{lstlisting} 
            
            Et le profil peut être également mis à jour de la manière qui suit : 
\begin{lstlisting}[style=api] 
  http://feedny.io/api/profiles/update/profile=username::preferences+algeria
\end{lstlisting}            
        \end{enumerate} 

    \subsection{Développement Front-end}
    Dans le monde du développement web et mobile, nous recherchons toujours des cycles de développement très courts, des délais de déploiement réduits avec une meilleure performance.

    Une technologie très récente (2016) se trouve au beau milieu de ces exigences, le développement d'application mobile hybride, en utilisant des techniques et des langages de programmation très répandus parmi les développeurs web (comme JavaScript ou HTML5 et CSS) enveloppées dans un Framework lui permettant de fonctionner nativement sur n'importe quel appareil et système d'exploitation (Androïd ou IOS).

    Ils existent plusieurs Frameworks d'applications mobiles hybrides, mais React-native \autoref{} développé par \emph{Facebook}, et jusqu'à l'écriture de ces lignes, reste le plus performant et le plus évolutifs tout en restant stable. On peut cités quelques points forts de React-native : 
        \begin{itemize}
            \item Open source, communauté ne cessent de s'agrandir,
            \item Facebook continue à investir dans sa croissance,
            \item Les composants réutilisables,
            \item Compatibilité avec les APIs, les SGBDs et les extensions tiers,
            \item Moins d'utilisation de la mémoire,
            \item ...
        \end{itemize}
    Tout ces arguments nous ont poussé à choisir React-native comme Framework pour développer notre application mobile. 

    \subsection{Fonctionnalités disponibles}
    Deux méthode d'utilisation de notre application sont disponible: Avec authentification, Sans authentification. La personnalisation des recommandations en dépendra.
    \begin{enumerate}[leftmargin=*]
        \item\textbf{Avec authentification (personnalisé)}\\
        Si l'utilisateur s'authentifie, la version personnalisée de notre application lui sera accessible et pourra dès la première connexion choisir ses catégories et ses sources favorites afin de construire son profil utilisateur.

        À partir de là, les articles proposés sont choisis en fonction de ses préférences, et les recommandations seront de plus en plus raffinées avec l'utilisation de l'application.

        Les utilisateurs de "Feedny" auront comme fonctionnalités:  
        \begin{itemize}
            \item \textbf{Catégorisation automatique d'articles :}\\
            Les articles sont classifiés dans différentes catégories suivant le sujet traité.
            \item \textbf{Résumé automatique :}\\
            Au lieu de lire un article complet, l'utilisateur aura un résumé généré automatiquement à partir de l'article qui lui permettra de retrouver tout les faits importants décrit dans l'article.
            \item \textbf{Traduction automatique :}\\
            l'utilisateur aura la possibilité d'avoir une traduction de l'article en cours de lecture.
            \item \textbf{Favoriser un article, une revue ou une catégorie :}\\
            L'utilisateur pourra aussi marquer un article, une source ou une catégorie précise d'articles comme favorite.
            \item \textbf{Mentionner la satisfaction :}\\
            Il aura la possibilité d'ajouter une mention de préférence "J'aime" ou "Je n'aime pas" sur la recommandation.
            \item \textbf{Recherche d'articles :}\\
            l'utilisateur aura la possibilité de faire une recherche spécifique des articles, et ceux en se basant soit sur : une recherche basé mots clés, une recherche basé sur une catégorie ou une recherche selon la source de l'article. 
            \item \textbf{Consultations des articles recommandés :}\\
            L'interface offrira la possibilité de consulter tout les détails concernant l'article (auteur, date de publication, etc.) 
        \end{itemize}
        Dès la connexion de l'utilisateur, ce dernier peut s'identifier en utilisant un nom d'utilisateur et un mot de passe ; sinon il utilisera les services non personnalisés du système qui sont présentés dans le point suivant.\\

        \item\textbf{Sans authentification (non personnalisé}\\
        Dans le cas où l'utilisateur ne souhaite s'identifier, la version non personnalisé sera entièrement accessible et il pourra :
        \begin{itemize}
            \item \textbf{Recommandation basée similarité :}\\
            Dans le cas non personnalisé, la recommandation se basera uniquement sur l'article consulté et lu et les différents articles disponible dans la base de données.    
            \item \textbf{Consultation des articles suggérés :}\\
            l'utilisateur pourra en effet au moment même ou il lit un article, d'avoir une suggestion d'autres articles jugés similaires par le système.
            \item Toutes les autres fonctionnalités de la version Personnalisé qui n'utilisent pas le profile utilisateur sont également disponible.
        \end{itemize}
    \end{enumerate}
    \vspace*{0.7cm}
    Nous allons voir maintenant, les différentes interfaces de noter application.
    \subsubsection{Authentification}
    L'utilisateur peut créer un compte afin de s'authentifier, il peut s'inscrire en utilisant son compte \emph{Google}, \emph{Facebook} ou \emph{Twitter}. 
        \begin{figure}[H]
            \centering
            \includegraphics[height=200pt,width=100pt]{img/chapter4/feedny/feedny.png}
            \caption{Espace d'authentification de "Feedy"}
            \label{}
        \end{figure}

    % \subsubsection{Interface principale}
    % L'espace principale de "Feedny" contient plusieurs 
    % \subsubsection{Favoris}
    % \subsubsection{Catégories}
    % \subsubsection{Article}