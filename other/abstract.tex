%!TEX program=luatex
\begin{center}
    \Large 
    \textbf{Résumé}
\end{center}
\setlength{\parindent}{0.5cm}
Avec l'émergence des TIC et la démocratisation de la production de contenu sur internet, on assiste à un tsunami informationnel. Ce dernier est très difficile à gérer manuellement, et même les outils informatiques classiques peinent à offrir des résultats concluants. Les journaux et les sites d'information contribuent pleinement à ce contenu, ce qui rend la revue de presse quotidienne dure à appréhender. 

Notre projet, propose un outil sophistiqué qui prend en considération les centres d'intérêt de chaque utilisateur, afin de lui suggérer les articles les plus pertinents de la presse algérienne et internationale, et ce dans les deux langues (Arabe et Anglais). Ceci est accompli en utilisant des techniques d'intelligence artificielle, de traitement automatique du langage et d'apprentissage automatique. 

\noindent
\textbf{Mots clés :} Traitement automatique du langage naturel, catégorisation de textes, corpus de résumé
automatique, résumé automatique, traduction automatique, profilage d'utilisateurs, recommandation d'articles de presse.  

\vspace*{0.8cm}

\begin{center}
    \Large 
    \begin{arab}
    ملخص
    \end{arab}
\end{center}
\begin{arab}
مع بروز تكنولوجيا الإعلام و الإتصال و دمقرطة محتوى صفحات الأنترنت، أصبح التحكم في الكم الهائل من المعلومات صعبا للغاية، سواءا يدويا أو حتى باستعمال أنظمة معلوماتية. وتساهم الصحف و المواقع الإخبارية بشكل مباشر في تزايد المحتوى، مما يجعل الإطلاع على الأخبار أمرا متعبا يتطلب الكثير من الوقت. 

لمعالجة هذه الإشكالية، نقترح في مشروعنا تطبيقا يأخذ بعين الإعتبار اههتمامات المستخدمين بهدف اقتراح مقالات من الصحافة الجزائرية و الدولية باللغتين العربية و الإنجليزية تلبي رغبة القارئ، و هذا باستخدام تقنيات الذكاء الإصطناعي و المعالجة الآلية للنصوص.

\textbf{الكلمات الدالة :} المعالجة الآلية للغات،  تصنيف النصوص، تلخيص النصوص، الترجمة الآلية، التوصية.  
\end{arab}

\vspace*{0.8cm}


\begin{center}
    \Large 
    \textbf{Abstract}
\end{center}

The Internet plays an increasingly important part in our daily lives as a source of written content for news and leisure. Yet it is tedious and difficult to sort through this staggering flow of information and stay updated with changes in our world, even using automated tools. 

Reading magazines and newspapers is too time-consuming, and there is a huge amount of online content that is updated or generated each minute. 

Our solution considers each user's interests and leverages Artificial Intelligence, Machine Learning and Natural Language Processing in order to suggest to relevant articles from the internet.

\noindent
\textbf{Keywords :} Natural Language Processing, Text classification, Automatic summarization corpora, Automatic summarization, Machine translation, User profiling, News recommendation. 