%!TEX program=luatex
\begin{center}
    \Large 
    \textbf{Résumé}
\end{center}

Avec l'émergence des TIC et la démocratisation de la production de contenu sur internet, on assiste à un tsunami informationnel. Ce dernier est très difficile à gérer manuellement, et même les outils informatiques classiques peinent à offrir des résultats concluants. Les journaux et les sites d'information contribuent pleinement à ce contenu, ce qui rend la revue de presse quotidienne dure à appréhender. 

Notre projet, propose un outil sophistiqué qui prend en considération les centres d’intérêt de chaque utilisateur, afin de lui suggérer les articles les plus pertinents de la presse algérienne et internationale, et ce dans les deux langues (Arabe et Anglais). Ceci est accompli en utilisant des techniques d'intelligence artificielle, de traitement automatique du langage et d'apprentissage automatique. 

\noindent
\textbf{Mots clés :} Traitement automatique du langage naturel, catégorisation de textes, corpus de résumé
automatique, résumé automatique, traduction automatique, profilage d'utilisateurs, recommandation d'articles de presse.  

\vspace*{0.8cm}

\begin{center}
    \Large 
    \begin{arab}
    ملخص
    \end{arab}
\end{center}
\begin{arab}
مع بروز تكنولوجيا الإعلام و الإتصال و دمقرطة محتوى صفحات الأنترنت، أصبح التحكم في الكم الهائل من المعلومات صعبا للغاية، سواءا يدويا أو حتى باستعمال أنظمة معلوماتية. وتساهم الصحف و المواقع الإخبارية بشكل مباشر في تزايد المحتوى، مما يجعل الإطلاع على الأخبار أمرا متعبا يتطلب الكثير من الوقت. 

لمعالجة هذه الإشكالية، نقترح في مشروعنا تطبيقا يأخذ بعين الإعتبار اههتمامات المستخدمين بهدف اقتراح مقالات من الصحافة الجزائرية و الدولية باللغتين العربية و الإنجليزية تلبي رغبة القارئ، و هذا باستخدام تقنيات الذكاء الإصطناعي و المعالجة الآلية للنصوص.

\textbf{الكلمات الدالة :} المعالجة الآلية للغات،  تصنيف النصوص، تلخيص النصوص، الترجمة الآلية، التوصية.  
\end{arab}

\vspace*{0.8cm}


\begin{center}
    \Large 
    \textbf{Abstract}
\end{center}

With the internet playing such an important part in our lives, especially in the way we communicate through it (social media, blogs, etc), it has become very difficult to control the flow of information, whether manually or even by using IT solutions.

Reading magazines and newspapers has become very time consuming, considering the huge amount of online content and the speed with which this content is updated.

Our solution considers each user's centers of interest in order to suggest to him/her articles from the internet, articles that are relevant to each one of them. For this purpose we use Artificial Intelligence, Machine Learning and Natural Language Processing.

\noindent
\textbf{Keywords :} Natural Language Processing, Text classification, Automatic summarization corpora, automatic summarization, Machine translation, user profiling, news recommendation. 